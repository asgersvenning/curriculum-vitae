\begin{rubric}{Education}
\entry*[2023 -- \ldots]
    \textbf{Ph.D.~Student, Section for Biodiversity, Department of Ecoscience, Aarhus University}\par
    \emph{Scalable Biodiversity Monitoring \& Analysis with Computer Vision}\par
    \pind Developing automated computer vision techniques for identifying and mapping insect and pollinator species from images. This project merges machine learning with biological research, aiming to enhance biodiversity monitoring by creating deployable models for species identification and ecological analysis. Focuses on bridging the gap between ecology and technology to advance the understanding of biodiversity changes, with a special interest in the ecological significance of pollinators.
%
\entry*[2022 -- 2025]%
	\textbf{M.Sc.~Biology, Aarhus University}\par
    \pind Courses in Computer Vision \& Deep Learning, Population Genetics \& General Biology. I conducted two large joint student research projects in Computer Science and Applied Biology on the use of image slicing for object detection on timelapse imagery using deep learning. The developed method was then applied for inference on flower detection, phenology classification and tracking. Lastly the resulting prediction were used to analyse the connection between climate/weather and flowering phenology.
%
\entry*[2019 -- 2022]%
	\textbf{B.Sc.~Biology, Aarhus University} in Functional Macroecology \& Biodiversity\par
	\emph{A Functional Investigation of Species Richness Relationships in the North American Tree Flora}\par
    \pind Quantifying the biases in topology indices of functional trait distribution in North American tree assemblages, I revealed the detrimental impact of using presence/absence data compared to abundance data. This analysis was conducted on a vast dataset encompassing over 20 million trees, 400 species, and 50 functional traits and utilized both model selection and generalized additive mixed models for robust semi-parametric hypothesis-testing.
%
\entry*[2015 -- 2018]%
    \textbf{Studentereksamen, Aarhus Katedralskole}, major in Mathematics \& Physics\par
%
\end{rubric}