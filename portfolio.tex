
\begin{rubric}{Project Portfolio}
\entry*[\hfill April 3\rd,\hspace{0.5em} 2025]%
    \textbf{\texttt{mini\textunderscore trainer}: A \texttt{Python} package for training deep learning image classification models}\par
    \href{https://github.com/asgersvenning/mini_trainer}{\color{blue}\textit{https://github.com/asgersvenning/mini\textunderscore trainer}}\par
    \pind My personal \texttt{Python} deep learning for image classification training with \texttt{PyTorch}. The module is designed using the "\textit{Factory method pattern}" to enable full extendability, with good default behaviour for all parts of training that are not relevant to a given research question.

\entry*[\hfill May 30\th,\hspace{0.5em} 2024]%
    \textbf{Computer Vision Pipeline for ETH BiodivX in the XPRIZE Rainforest Finals}\par
    \href{https://github.com/asgersvenning/BiodivX-XPRIZE-ML-Pipeline}{\color{blue}\textit{https://github.com/asgersvenning/BiodivX-XPRIZE-ML-Pipeline}}\par
    \pind The overall computer vision pipeline of the ETH BiodivX team for the XPRIZE Rainforest competition finals.

\entry*[\hfill November 8\th,\hspace{0.5em} 2023]%
    \textbf{\texttt{pyRemoteData}: A \texttt{Python} package for High-Latency High-Bandwidth data streaming with \texttt{lftp}}\par
    \href{https://github.com/asgersvenning/pyremotedata}{\color{blue}\textit{https://github.com/asgersvenning/pyremotedata}}\par
    \pind \texttt{pyRemoteData} is a module developed for scientific computation using the remote storage platform \href{https://erda.au.dk/}{ERDA} (Electronic Research Data Archive) provided by Aarhus University IT. To facility high-throughput computation in diverse settings, \mbox{\texttt{pyRemoteData}} handles data transfer with multithreading and asynchronous data streaming using thread-safe buffers.

\entry*[\hfill May 23\rd,\hspace{0.5em} 2023]%
    \textbf{Population Genetics on Genomes: Exploring non-African archaic segments}\par
    \href{https://asgersvenning.com/exploring-non-african-archaic-segments}{\color{blue}\textit{https://asgersvenning.com/exploring-non-african-archaic-segments}}\par
    \pind My final project in the masters level course "Population Genetics on Genomes" at the Department of Bioinformatics at Aarhus University. In collaboration with then Bioinformatics masters student Bjarke Meyer Pedersen (now Ph.D. student) I investigated the whole-genome archaic ancestry relationships between 358 individuals from West-Eurasia, South-Asia, East-Asia, Central-Asia-Siberia \& Melanesia.

\entry*[December 9\th,\hspace{0.5em} 2022]%
    \textbf{Using Timelapse Imagery to Describe High-Resolution Phenology Patterns in \textit{Silene acaulis} \underline{\&} YOLOFlower: Near-Real Time Flowering Stage Detection}\par
    \href{https://asgersvenning.com/timelapse-phenology-silene-acaulis}{\color{blue}\textit{https://asgersvenning.com/timelapse-phenology-silene-acaulis}}\par
    \pind A joint computer science and ecology project with my now Ph.D. supervisor Toke T. Høje and then Computer Science masters student Simon Sataa-Yu Larsen on flower detection and tracking in timelapse images with computer vision and the ecological inferences it enables.

\entry*[September 7\th,\hspace{0.5em} 2022]%
    \textbf{Learning tool for the course ‘Dansk Flora og Vegetationsøkologi’}\par
    \href{https://github.com/asgersvenning/Dansk-Flora-App}{\color{blue}\textit{https://github.com/asgersvenning/Dansk-Flora-App}}\par
    \href{https://vegdyn.au.dk/learn}{\color{SwishLineColour}\textit{https://vegdyn.au.dk/learn}}\par
    \pind An application for practicing species identification and habitat knowledge in Denmark. I built the app to practice for the exam in the bachelor course "Dansk Flora og Vegetationsøkologi" (Danish Flora and Vegetation Ecology), and later developed it further for general use by the student body in later iterations of the course as a contractor project under Professor Signe Normand and Data Scientist Derek Corcoran Barrios.

% \entry*[October 4\th,\hspace{0.5em} 2021]%
%     \textbf{Effects of sand deposition on vegetation composition in tidal flat ecosystems with a focus on cordgrass dominance}\par
%     \href{https://asgersvenning.com/sand-deposition-tidal-flat-spartina}{\color{blue}\textit{https://asgersvenning.com/sand-deposition-tidal-flat-spartina}}\par
%     \pind 

\end{rubric}